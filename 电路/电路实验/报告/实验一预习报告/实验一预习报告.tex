\documentclass{ctexart}
\usepackage{graphicx}
\begin{document}
\begin{center}
    \LARGE{电路实验第一课 预习报告}
\end{center}
\begin{center}
    Leo
\end{center}
\section{实验要求}
\begin{enumerate}
    \item 认识基本的电路元件,包括分类、参数、用途等
    学习常见实验仪器的测量对象、范围、精度、用途等
    \item 掌握其使用与调节方法与常见故障的排除方法
\end{enumerate}

\section{实验原理}
\subsection{示波器的作用}
可以描绘波形曲线,用来观察电压信号的幅度、频率和相位等信息
\subsection{怎样调节旋钮,才能看到完整的波形?}
波形位置偏高或偏低:用垂直旋钮调节。

波长超出屏幕或太短:调节扫描速率旋钮至屏幕上显示1-2个周期的波形。

波形幅度不合适:调节相应通道的灵敏度,尽量使整个波形铺满屏幕。
\subsection{DC耦合、AC耦合与接地的区别与使用场景}
DC耦合:即直流耦合,不隔绝输入信号中的直流分量。测量脉冲信号的高低电平、峰峰值、观测直流信号或低频信号时用。

AC耦合:即交流耦合,隔绝直流分量。当信号频率高于几十赫兹或交流分量需要被放大时使用。

接地耦合:测量对地电压。可用于排除探头故障,还可以用于实验开始前的电平校准。
\subsection{示波器探头的作用,倍率开关的区别}
作用:将待测信号引入示波器;抑制外界信号的干扰。

$10\times $开关用于观测高频信号或宽带信号,让整个频段有平坦的10倍衰减特性。此时测量的的电压数值要乘10.当屏幕里选择了探头$\times 10 $选项,则测量电压的结果要除以10
\subsection{示波器触发设置的作用,怎样保持波形稳定}
当输入的波形中某一个波形满足人为设定的条件时,示波器可以捕获该波形及其邻近部分并显示出来。只有稳定的触发才会有稳定的显示。因为每次扫描或采集都从输入信号上与用户定义的触发条件开始,所以捕获的波形应该是相重叠的,从而显示稳定的波形。

一般把触发方式设置为Auto。还要设置一个触发电平线。

\subsection{测量电压与时间有几种方法?}
读格数:时间=格数*时基(或电压)档位

自动测量

光标测量
\subsection{测量信号周期的方法?}
数格子,再乘以时基得到周期,也可以用自动(或光标)测量得到时间再相减。
\subsection{测量高低电平的方法?}
可以使用测量系统,直接读取top和base的值;可以用光标测量,在t-Y显示下读取屏幕右侧的$Y_1$和$Y_2$值.

\subsection{DDS信号源的作用、输出波形?如何调节电压值和频率值?}
作用:为被测电路提供所需的信号

输出波形:正弦波、方波、三角波、脉冲、高斯噪声、DC和任意波。

调节参数:按Parameter键可以直接进入参数设置界面调节电压与频率,可以用数字键盘也可以用旋钮。
\section{其他实验基础知识}
\subsection{常见的脉冲信号}

\begin{enumerate}
    \item 正弦波
    \item 方波
    \item 三角波
\end{enumerate}


\section{实验仪器与器材}
\subsection{脉冲信号的参数及其表示}
\begin{enumerate}
    \item 周期T:相邻的正脉宽和负脉宽时间长度的和
    \item 幅度Vm:高电平和低电平之差
    \item 占空比:正脉宽/周期*100%
    \item 峰峰值$V_{P-P}$:(正弦波中) ,波峰和波谷幅值的差
    \item 有效值$V_{RMS}$: (正弦波中),峰峰值除以2$\sqrt{2}$
    \item 直流分量:设正弦函数:$V=A+V_P \times sin \omega t $,则直流分量为 A
\end{enumerate}

\section{实验仪器}
\subsection{示波器}
\begin{enumerate}
    \item 水平控制:调节触发点位置和时基
    \item 垂直控制:调节波形的垂直位移和电压度量单位(可粗调可细调),有两个通道(以颜色区分),可以打开波形运算菜单
    \item 触发控制:一般用Auto模式
    \item 运行控制:可打开自动显示功能,可设置运行或关闭状态
    \item 多功能按钮:用于选择子菜单之类的东西
    \item 功能菜单:光标模式、测量系统、采样设置菜单和文件存储与导出
    \item 与探头配合使用
\end{enumerate}

\subsection{探头}
分为接地端和BNC端

有99\%的问题出在探头上:可能是没有接地,可能是探头损坏,可能是与示波器接触不良等等

\subsection{DDS信号源}
基本参数:频率范围、幅度范围

常用功能
\begin{enumerate}
    \item Waveforms:选择波形
    \item Parameter:直接设置基本波形参数
    \item Ch1/Ch2:切花通道
    \item 通道输出控制:控制信号输出与否
\end{enumerate}
\subsection{直流电源}
有三组独立输出:两组可调电压和一组固定可选的电压值,还有三种输出模式:独立、并联和串联
\begin{enumerate}
    \item NO.1-5:选择存储位置
    \item SER:设置CH1/Ch2的串联模式
    \item PARA:设置CH1/Ch2的并联模式
    \item RECALL:存储系统调出参数设置
    \item SAVE:保存参数
    \item LOCK:长按开启或关闭锁键,防止误触
    \item FINE:细调,参数变为以最小步长变化
\end{enumerate}
\subsection{色环电阻}
分为四环电阻与五环电阻。

都以环较为密集的一边为左边。四环电阻从左到右四个环分别为十位、个位、放大倍数、误差范围;五环电阻从左到右为百位、十位、个位、倍数、误差范围。
% \section{实验表格}
% \subsection{电阻、电容参数测量}
% \begin{table}[!ht]
%     \centering
%     \caption{电阻参数测量}
%     \setlength{\tabcolsep}{20mm}%7可随机设置,调整到适合自己的大小为止
%     \begin{tabular}{|c|c|c|}
%     \hline
%         测量值  & ~ & ~ \\ \hline
%         色环  & ~ & ~ \\ \hline
%         标称阻值  & ~ & ~ \\ \hline
%         标注误差  & ~ & ~ \\ \hline
%         实测误差 & ~ & ~ \\ \hline
%     \end{tabular}
% \end{table}


% \begin{table}[!ht]
%     \centering
%     \caption{电容的测量}
%     \setlength{\tabcolsep}{20mm}%7可随机设置,调整到适合自己的大小为止
%     \begin{tabular}{|c|l|l|}
%     \hline
%         标称容量  & ~ & ~ \\ \hline
%         万用表测量电容量  & ~ & ~ \\ \hline
%         实测误差 & ~ & ~ \\ \hline
%     \end{tabular}
% \end{table}
\end{document}
