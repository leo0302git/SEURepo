\chapter{指令系统}
\section{寻址方式}
\subsection{立即寻址}
操作数包含在指令中,它是一个8或16位常数,称为立即数。

示例
\begin{lstlisting}
    MOV AX,1860H
\end{lstlisting}
表示将1860H送入AX寄存器中。其中,立即数在代码段中,紧跟在操作码之后。
\subsection{寄存器寻址}
操作数包含在寄存器中,由指令指定寄存器名
\begin{lstlisting}
    MOV AX,BX
    MOV CH,AL
\end{lstlisting}
表示将寄存器BX中的数送入AX寄存器中。第二条指令说明低位和高位可以互相传数字
\subsection{存储器寻址}
操作数在存储器中。这种方式速度慢,表达方式最多,情况最复杂,要求能在1MB存储空间中寻址。段内偏移量也称有效地址EA。
\subsubsection{物理地址计算方法}
图示如下!!!(存储器地址计算图示)
段寄存器可以是CS,SS,DS,ES;有效地址EA可以是
\begin{enumerate}
    \item 指令中直接指定的16位直接位移量,要加方括号
    \item BX/BP/SI/DI,{\color{red}{出现BP(不管是源处还是目的处)时默认使用SS提供段基址,但允许使用段超越前缀将SS修改为CS/DS/E,没有BP的,默认使用DS}}
    \item 位移量+基址/变址
    \item [BX/BP]+[SI/DI]+位移量,{\color{red}{画斜线的两者不能同时出现}}
\end{enumerate}
EA的24种表示法!!!图

\subsection{其他}
\subsubsection{隐含寻址}
指令中不明确指明操作数,但有隐含规定的 寻址方式。如:DAA对AL中的数进行十进制调整;CLI对中断标志清零
\subsubsection{一条指令包含几种寻址方式}
例如
\begin{lstlisting}
    MUL DL 
    ## AX<-AL*DL,源操作数为寄存器和隐含的寄存器AL
\end{lstlisting}
\subsubsection{IO端口寻址}
直接端口寻址:端口地址由指令直接给出,范围是00-FFH,若超出要先放到寄存器里。如
\begin{lstlisting}
    IN AL,40H
    OUT 83H,AL
\end{lstlisting}
间接端口寻址:端口地址由DX寄存器提供,其范围为0000-FFFFH
\begin{lstlisting}
    MOV DX,2F0H
    IN AL,DX ##2F0H读入到AL
    MOV DX,300H
    OUT DX,AL ##AL输出到300H
\end{lstlisting}
\subsubsection{转移类指令寻址}
详见控制转移指令
\section{数据传送}
\subsection{通用传送}
\subsubsection{MOV}
将操作数(字或字节)传送到目标操作数,源可为8位/16位通用寄存器(包括AX-DX,SI,DI,BP,SP),段寄存器(CS不能作为目的操作数),存储器或立即数。目的操作数不能为立即数,其他同源操作数。
四种数据来源的传送关系如图!!!
下面是一些示例
\begin{lstlisting}
    # 立即数送存储器
    MOV [BX],0FFH
    # 立即数送通用存储器
    MOV AL,30H
    MOV BL,'$'
    MOV AX,1250H
    MOV BX,OFFSET TABLE
    # 存储器和通用寄存器相互传送
    MOV [BP],BX # 带方框表示是段加偏移对应的存储器单元而不是BP本身的值受到改变
    MOV CL,5[BX]
    # 存储器与段寄存器之间相互传送
    MOV [SI],DS
    # 段寄存器和通用寄存器相互传送
    MOV ES,AX
    # 通用寄存器之间相互传送
    MOV AX,DX
\end{lstlisting}
常见错误
\begin{lstlisting}
    # 立即数不能作目的地
    MOV 60H,AL
    # 存储器之间不能直接传送
    MOV [BX],[SI]
    # CS不能作目的操作数
    MOV CS,AX
    # 通用寄存器无IP寄存器
    MOV BX,IP
    # 段寄存器之间不能直接传送
    MOV DS,ES 
    # 数的长度不一致
    MOV CX,AL
\end{lstlisting}
\subsubsection{PUSH}
将源操作数入栈并使SP减二.{\color{red}低位先入栈}
\begin{lstlisting}
    PUSH AX # 把AX入栈
\end{lstlisting}
\subsubsection{POP}
将当前SP指向的栈顶的一个字送到目的操作数中,并使SP+2.操作数可以是16位通用寄存器,DS/SS/ES或存储单元。{\color{red}堆栈操作总是以字为单位}
\begin{lstlisting}
    POP AX # 先出栈的一个字节给到低位
\end{lstlisting}
\subsubsection{XCHG}
把字或字节的源操作数和目的操作数交换。可以发生在寄存之间或寄存器与存储器之间,但是段寄存器不能作操作数。
\begin{lstlisting}
    # Right
    XCHG AX,BX
    XCHG [BI],CL 
    # Wrong
    XCHG AX,0AF4H #立即数不能交换
    XCHG SS,[SI] # 段寄存器不参与交换
    XCHG [SI],[BX+1] # 存储器之间不能直接交换
\end{lstlisting}
\subsubsection{XLAT}
将一个字节从一种代码转换成另一种代码。注意:
\begin{enumerate}
    \item 使用指令前必须先创建一个表格,将转换表的起始地址装入BX中
    \item AL存表头地址到所查找的某一项之间的位移量,根据位移量从表中查到转换后的代码值送入AL
    \item 表中最多存256字节
\end{enumerate}
典型的例子是用该指令将十进制数转换成七段显示管的代码
\begin{lstlisting}
    TABLE: DB 40H,79H,24H,30H,19H
           DB 12H,02H,78H,00H,18H # 做表
           ... 
        LEA BX,TABLE # 把表头地址给BX
        MOV AL,5 # 把表内的偏移量给AL
        XLAT # 此时AL中有‘5’的七段代码12H(表内从零开始数)
\end{lstlisting}
\subsection{输入输出}
\subsubsection{IN(OUT同理)}
把指定端口的内容读到AL(for byte)或AX(for word)中
\begin{lstlisting}
    # 格式1:立即数指示端口
    IN AL,nn #nn指代的8位端口的内容读到AL,nn=00-FFH
    IN AX,mm #mm指代的16位端口读到AL,mm+1指代的16位端口读到AH.mm=00-FEH
    # 格式2:用寄存器给端口号
    IN AL,DX # DX=0000-FFFFH
    IN AX,DX # DX=0000-FFFEH DX指代的16位端口读到AL,DX+1指代的16位端口读到AH
\end{lstlisting}
{\color{red}{当口地址大于FFH时,必须用格式2}}。典型示例:扬声器发声。
\subsection{地址目标}
这是一类专门用来传送地址码的指令
\subsubsection{LEA}
取源操作数地址的偏移量,传送到目的操作数所在的单元(load effective address)。{\color{red}源操作数必须是存储单元,目的操作数必须是除段寄存器以外的16位寄存器}
\begin{lstlisting}
    # 注意区分LEA(取偏移地址)和MOV(取内容)
    # 设SI=1000,DS=2000,(21000H)=1234,则
    LEA BX,[SI] # BX=1000
    MOV BX,[SI] # BX=1234
    # 下面两段代码等效
    LEA BX,TABLE
    MOV BX,OFFSET TABLE
\end{lstlisting}
\subsubsection{LDS}
load DS:从源操作数指定的存储单元中取出4个字节的地址指针送进一对目的寄存器中(常用SI寄存器,但是不能是段寄存器)
load ES:与LDS基本相同但是后两个字节给到ES而非DS,目的操作数也常为DI而非SI
\begin{lstlisting}
    # 设DS=0100,BX=0020,(01020H)=26A0H,(01022H)=B500H
    LDS SI,[BX] # 低位两个字节给SI,SI=26A0,高位两字节给DS,DS=B500
    LES DI,[BX] # DI=26A0,ES=B500
\end{lstlisting}
\subsection{标志传送}
\subsubsection{LAHF}
load AH flags:将SF,ZF,AF,PF,CF送到AH的7,6,4,2,0位,而5,3,1位为任意值。
\subsubsection{SAHF}
send AH flags:将AH的7,6,4,2,0位送到SF,ZF,AF,PF,CF,而高位flag(OF,DF,IF,TF)不受影响。
\subsubsection{PUSHF/POPF}
将整个标志寄存器的内容压栈/出栈,并使SP-2/+2(仍然是低位先入栈,高位先出栈)
\section{算数运算}
总结图!!!
\subsection{加法}
\subsubsection{ADD}
目的数被赋值为“目的数+源”
\begin{lstlisting}
    # RIGHT
    ADD AL,45H
    ADD BL,DL 
    ADD [BX],CL
    # Wrong
    ADD 85H,AL # 立即数不能被赋值
    ADD 5[BX],[BP] # 两个存储器内容不能直接相加
    ADD BX,CL # 字节和字不能相加
\end{lstlisting}
加法指令结束后将影响六个标志位,如何解释这些标志位(是无符号数相加还是有符号数相加决定是进位还是溢出)取决于程序员。
\subsubsection{ADC}
add carry:带进位的加法指令。目的=目的+源+CF
\subsubsection{INC}
increment:增量指令,对目的操作数加1.操作后影响AF,OF,PF,SF,ZF,但不影响CF
\begin{lstlisting}
    INC BL 
    INC BYTE PTR[BX] # 对内存字节单元内容+1
    INC WORD PTR[BX] # 对内存字单元内容+1
\end{lstlisting}
\subsubsection{AAA}
ascii-adjusted add:加法的ASCII调整指令。用ADD或ADC对两个非压缩十进制数或ASCII表示的十进制数作加法,且运算结果保留在AL中后,用AAA可以将运算结果调整为一位非压缩十进制数,仍然放在AL中。若AF=1,则表示高位有进位,进到AH中。

*非压缩十进数(BCD数):高四位全为零
例题图!!!
\subsubsection{DAA}
decimal-adjusted add:加法的十进制调整指令。将两个压缩的BCD数相加的结果调整为正确的压缩BCD数,相加结果必须在AL中才能使用DAA。
例题图!!!
\subsection{减法}
\subsection{乘法}
\subsection{除法}
\section{逻辑移位}
\subsection{逻辑运算}
\subsection{算数移位}
\subsection{循环移位}
\section{字符串}
\subsection{传送}
\subsection{比较}
\subsection{扫描}
\subsection{装入}
\subsection{存储}
\section{控制转移}
\subsection{无条件转移}
\subsection{过程调用}
\subsection{条件转移}
\subsection{条件循环控制}
\subsection{中断指令}
\section{控制指令}
\subsection{标志操作指令}
\subsection{外部同步指令}
\subsection{停机和空操作}