\chapter{中断}
\section{8086中断系统}
\subsection{中断概述}
CPU执行正常的程序时,由于某些事件产生,暂时中止当前程序的执行,转去执行为时间服务的“中断服务程序”,然后返回原程序继续执行,这个过程称为“中断”。中断的基本概念包括中断源、中断屏蔽、中断嵌套、中断矢量表和中断优先级。
\subsection{中断源}
\begin{equation}
    \text{中断源}
    \begin{cases}
        \text{硬件中断(外部中断)}
        \begin{cases}
            NMI\\
            INTR
        \end{cases}
        \\
        \text{软件中断(内部中断)}
        \begin{cases}
            INT \quad n\\
            INTO \\
            Division \quad error\\
            IF=1
        \end{cases}
    \end{cases}
\end{equation}
\subsection{硬件中断}
从8086/8088的NMI引脚引入的为不可屏蔽中断,如奇偶校验错或电源掉电。

从INTR引脚引入的为可屏蔽中断,中断源接到8259A的IR0-IR7引脚,经过8259A判优之后,选择优先级最高的中断向CPU提出中断请求。
\subsection{软件中断}
INT n(n=00-FFH)以指令形式安排在程序中。

INTO 溢出标志OF为1时,若再执行INTO指令,则自动产生溢出中断。

除法错中断:当除数为0或商超过寄存器的最大容纳值时产生该中断。

单步中断:IF=1时,CPU每执行一条指令后都产生单步中断,将各寄存器和有关寄存器的内容显示在CRT上。
\subsection{中断矢量表}
\begin{enumerate}
    \item 中断矢量表用于存放中断服务程序的入口地址
    \item 共有256种中断
    \item 每种中断的入口地址(CS:IP)占4个内存单元
    \item 共占用1K单元,位于000-3FFH范围内
\end{enumerate}
中断矢量表的格式如图!!!
\subsection{响应过程}
中断响应流程图如图所示。!!!
其中,中断处理的程序流程为
\begin{enumerate}
    \item 标志寄存器入栈
    \item IF——>TEMP
    \item 清除IF和TF:清除IF禁止其他可屏蔽中断进入,清除TF禁止单步运行行模式。
    \item CS和IP入栈
    \item 进入中断处理程序
    \item 检查NMI,若有,重新第一步,若无,STI打开中断,允许其他中断进入
    \item 检查TEMP=1?若是,重新第1步,若否,执行中断处理程序
    \item 用IRET弹出IP、CS、标志寄存器
    \item 返回断点
\end{enumerate}
其中进入中断处理程序前都是由硬件自动完成的。中断可以嵌套,优先级从高到低为IR0,... IR7.
\section{8259A基本原理}
可编程中断控制器,允许8级硬件中断,级联后可以扩展到64级,每一级都可以允许或屏蔽中断。判别优先级后,选择高级中断类型号送给CPU
\subsection{引脚功能}
\begin{enumerate}
    \item $D_0-D_7$:数据总线,与系统DB或通过总线驱动器与系统DB连接
    \item $\overline{CS},\overline{WR},\overline{RD}$:片选、写、读信号
    \item $\overline{INTA}$:中断响应信号。CPU响应INTR中断后,发出两个连续的$\overline{INTA}$信号,第一个表示CPU已经开始处理,第二个发来后,8259从DB上送出中断类型号。
    \item $A_0$:端口选择线,0表示选中偶地址
    \item $CAS_0-CAS_2$:级联信号,主片的$CAS_0-CAS_2$做输出信号,从片的$CAS_0-CAS_2$做输入信号
    \item $\overline{SP}/\overline{EN}$:从设备编程/允许缓冲线
    \item $IR_7-IR_0$:中断设备的中断请求信号输入端,可以是电平也可以是上边沿触发
\end{enumerate}
8259A内部结构框图!!!
\begin{enumerate}
    \item 数据总线缓冲器:8位双向三态缓冲器,传送命令字(写入)、状态字(读出)、中断向量字(读出)
    \item 控制电路:有一组初始化命令字寄存器和一组操作命令字寄存器,按编程方式管理8259A工作
    \item 读写控制电路:
    \begin{table}{H}
        \begin{tabular}{|c|c|c|c|c|}
            $\overline{CS}$&$\overline{WR}$&$\overline{RD}$&$A_0$&读写功能\\\hline
            0&1&0&0&CPU往8259写ICW1,OCW2,OCW3\\ \hline
            0&1&0&1&CPU往8259写ICW2,ICW3,ICW4,OCW1\\\hline
            0&0&1&0&IRR/ISR向CPU传送数据\\\hline
            0&0&1&1&IMR向CPU传送数据\\\hline
            1&x&x&x&高阻态\\\hline
            x&1&1&x&高阻态\\\hline
        \end{tabular}
    \end{table}
    \item 级联缓冲器/比较器
    \item 中断服务寄存器ISR:存放正在处理中的中断请求信号。任一级中断被响应时,相应的$IS_n$位置为1,中断处理结束后清零。多重中断时可能会多位置一
    \item 优先权判别器PR:对IRR寄存器的中断请求进行判定,让优先级最高的中断服务送到ISR中
    \item 中断请求寄存器IRR:存放$IR_7-IR_0$上的中断请求信号,哪个引脚上有请求,相应位就置一。电平触发还是前边沿触发由编程确定。
    \item 中断屏蔽寄存器IMR:用软件使某位置一,则禁止对应位的中断请求信号进入PR和ISR
\end{enumerate}
\subsection{工作方式}
\subsubsection{设置优先级的方式}
\begin{enumerate}
    \item 完全嵌套方式:优先级固定,禁止同级中断
    \item 特殊全嵌套方式:优先级固定,$IR_0$最高,允许同级中断
    \item 优先级自动循环方式:某设备服务后,优先级排到最后,将最高权利赋给比他低一级的中断,初始$IR_0$最高
    \item 优先级特殊循环方式:同上,最低优先级由编程决定,如可以编程让$IR_4>\dots>IR_7>IR_0>\dots>IR_3$
\end{enumerate}
\subsubsection{屏蔽中断源的方式}
\begin{enumerate}
    \item 普通屏蔽:用OCW1对IMR置位为1来实现
    \item 特殊屏蔽方式:先通过OCW3使其工作在特殊屏蔽方式,再用OCW1使本级中断的屏蔽寄存器的相应位置一,仅屏蔽当前正在处理的这级中断本身,允许高级或低级中断进入
\end{enumerate}
\subsubsection{结束中断的方式}
\begin{enumerate}
    \item 自动结束中断AEOI:收到第二个$\overline{INTA}$信号后沿,发中断结束信号使$IS_n=0$,仅用于单片8259A场合。
    \item 一般中断结束命令EOI:收到CPU发来的EOI命令后,将ISR中优先级最高的$IS_n$复位,适用于完全嵌套方式
    \item 特殊中断结束命令SEOI:ISR不知道哪一级中断是最后处理的,需要OCW2的L2-L0字段来指出对哪个$IS_n$复位。适用于非完全嵌套方式(优先级循环方式)
\end{enumerate}
\subsubsection{系统总线的连接方式}
\begin{enumerate}
    \item 缓冲方式:8259A需要总线驱动器才能与DB连接
    \item 非缓冲方式:适用于单片或少量级联的系统
\end{enumerate}
\subsubsection{中断查询方式}
中断查询方式通常用在64级中断的场合。用OCW3使P=1,可以发出查询命令,再执行IN指令,读入状态字,了解中断状态。状态字格式为"$IxxxxW_1W_2W_3$",I=0表示无中断命令,3位W给出优先级最高的设备的编码,并使相应的$IS_n=1$