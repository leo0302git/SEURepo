\chapter{并行接口芯片8255A}
通用可编程并行IO接口芯片,内部有ABC3个数据端口,可工作于三种方式
\section{引脚简介}
\begin{enumerate}
    \item $D_0-D_7,\overline{CS},\overline{WR},\overline{RD}$:数据位和片选、读写控制信号
    \item $A_1A_0$:端口选择信号,从0到3分别表示选中A、B、C和控制字寄存器
    \item $PA_0-PA_7$:A口输入输出引线,与外设相连传送数据信息
    \item $PB_0-PB_7$:B口输入输出引线,与外设相连传送数据信息
    \item $PC_0-PC_7$:C口输入输出引线,与外设相连传送数据信息,可分为两组使用
\end{enumerate}
\section{内部结构}
如图所示!!!
\section{控制字}
控制字分为方式选择字和置位复位字。方式选择字定义各端口的工作方式,置位复位字使C口的任一位置位或复位。

方式选择字的格式
\begin{enumerate}
    \item D7=1:标志位,表示本字为方式选择字
    \item D6D5:指定A口方式,0为方式0,1为方式1,其余为方式2
    \item D4:决定A口I/O
    \item D3:C口高4位I/O
    \item D2:决定B口方式:0为方式0,1为方式1
    \item D1:B口I/O
    \item D0:C口低四位I/O
\end{enumerate}
置位复位字格式
\begin{enumerate}
    \item D7=0:标志位
    \item D6D5D4:任意,通常置为0
    \item D3D2D1:用二进制数的数值来选中$PC_0-PC_7$
    \item D0:指示所选中位想要置为1或0
\end{enumerate}
\begin{lstlisting}
    MOV AL,00000100B
    OUT 63H,AL ;控制口地址为63H,让PC2输出低电平
\end{lstlisting}
\section{工作方式}
\begin{enumerate}
    \item 方式0:基本输入输出方式:A、B、C、任意口都可以做输入输出,输出信号能锁存,输入不能锁存
    \item 方式1:选通输入输出方式:A、B做8位数据口输入输出,但要在联络信号的控制下才能完成IO。C口6根先做联络信号,余下两根传送数据。A、B的IO数据均能锁存
    \item 方式2:双向传送方式。仅A口可以工作在本方式,IO均能锁存但不能同时进行。C口的$PC_7-PC_3$做A口的联络控制信号,余下3根用来IO
\end{enumerate}