\chapter{汇编语言}
\section{基本概念}
汇编语言程序:用指令的助记符、符号地址、标号、伪指令等汇编语言编写的程序称为源程序,用.asm文件表示。源程序要经过汇编程序翻译为机器能认识的二进制码.exe文件后才能让机器执行。

基本流程:编辑程序--汇编程序--连接程序--得到可执行文件

汇编语言源程序格式:
\begin{enumerate}
    \item 指令性语句:“标号:指令助记符 操作数 ;注释”
    \item 指示性语句:名字 伪指令助记符 ;注释
\end{enumerate}
\section{表达式}
表达式由运算对象和运算符组成,运算结果做操作数使用,运算对象可以是常数、变量或标号。常用运算符如下表
\begin{table}[H]
    \begin{tabularx}{\textwidth}{|c|X|X|}%l小写X大写
        \hline
        类型 & 运算符和运算结果 & 举例 \\ \hline
        算数运算符 & +-*/MOD,SHL,SHR & MOV AL,32/5,汇编成MOV AL,6\\\hline
        逻辑运算符&AND,OR,XOR,NOT &MOV AL,30H OR,05H汇编成 MOV AL,35H\\\hline
        关系运算符& EQ,NE,LT,GT,LE,GE&MOV AX,15 EQ 0FH汇编成MOV AX,0FFFFH(因为15等于0FH)\\\hline
        数值返回&OFFSET,SEG,TYPE(返回字节数),LENGTH(返回单元数),SIZE(返回总字节数)& \\\hline
        修改属性&段超越前缀,PTR修改类型属性,HIGH/LOW分离高低字节,SHORT短转移 &HIGH 1234=12H\\\hline
        其他&圆括号改变优先级,方括号间接寻址或下标 & \\ \hline
    \end{tabularx}
\end{table}
\section{伪指令}
伪指令为指示性语句,没有对应的机器码,不占用内存单元,仅起说明作用。这里列出了常用的伪指令。
\begin{table}[H]
    \begin{tabularx}{\textwidth}{|c|X|X|}%l小写X大写
        \hline
        伪指令语句 & 说明 & 举例\\ \hline
        等值伪指令EQU & 用于给常数、变量和表达式定义一个符号名 & COUNT EQU 100\\ \hline
        段定义语句SEGMENT...ENDS& ~ & ~\\ \hline
        段分配语句ASSUME& 任何时候都只允许四个段同时存在,该语句用来确定哪些段做DS,CS,ES,SS。CS必须要有& ~\\ \hline
        过程定义语句PROC...ENDP&定义一些子程序供主程序调用& PRO1 PROC FAR ... RET PRO1 ENDP\\ \hline
        程序结束语句END&放在源程序最后一行,汇编程序遇到END语句就停止进行& \\ \hline
    \end{tabularx}
\end{table}
\section{DOS功能调用}
DOS相当于一些提前写好的子程序,让用户可以方便地与设备和接口打交道。

调用方法:
\begin{enumerate}
    \item 将功能号放入AH中
    \item 入口参数放入指定寄存器中(可选)
    \item 执行INT 21H
    \item 调用结束后得到出口参数,或显示在CRT上
\end{enumerate}
\subsection{键盘功能调用}
\subsubsection{功能号=1、6、7、8}
等待键盘,有键压下时,将该键的ASCII码存入AL。区别在于
\begin{enumerate}
    \item 1,6号功能调用会回显该字符;7,8号不回显
    \item 1号功能调用还要检查是否按下ctrl+break键,若是,则执行INT 23H中止程序。
\end{enumerate}
\subsubsection{功能号=0AH}
键入字符串,将其存入显示缓冲区。该缓冲区由DS:DX指向。
\begin{lstlisting}
    DATA SEGMENT ;定义显示缓冲区
    BUF DB 50 ;最大字节数为50
        DB ?;用来存储实际键入的字节数
        DB 50 DUP(?);存键入字符的ASCII码
    DATA ENDS
    ... 
    CODE SEGMENT 
    MOV AX,DATA 
    MOV DS,AX ;DS指向缓冲区段址
    MOV DX,OFFEST BUF ;DX指向缓冲区偏移地址
    MOV AH,0AH;调用DOS功能
    INT 21H
    ... 
    CODE ENDS
    ;缓冲区的信息为32 实际键入数 键入字符的ASCII码 0D(回车符) 0 0 0 0 (后续未用到的地方填0)
\end{lstlisting}
\subsection{显示功能调用}
\subsubsection{功能号=2}
显示单个字符,入口参数为DL,其中存储要显示的字符的ASCII码
\begin{lstlisting}
    MOV DL,0EAH
    MOV AH,2
    INT 21H
\end{lstlisting}
\subsubsection{功能号=9}
显示以\$为结尾的字符串(\$符号不显示),字符串起始地址为DS:DX,作为入口参数。
\begin{lstlisting}
    DATA SEGMENT 
    MESS DB 'TRY AGAIN!',0DH,0AH,'$'
    DATA ENDS 
    CODE SEGMENT 
    ... 
    MOV SEG MESS
    MOV DS,AX
    MOV DX,OFFSET MESS 
    MOV AH,9
    INT 21H
    ... 
    CODE ENDS  
\end{lstlisting}
\subsection{打印功能调用}
功能号=5,入口参数为DL,里面存放要打印的字符
\begin{lstlisting}
PRNT DB 'YOU ARE RIGHT!'
COUNT EQU $-PRNT;表示将当前地址的值减去PRNT的地址,实际上就是PRNT的长度
        ... 
        MOV CX,COUNT 
        MOV BX,0
 NEXT:  MOV AH,5
        MOV DL,PRNT[BX]
        INT 21H
        INC BX 
        LOOP NEXT
\end{lstlisting}
\subsection{日期时间调用}
\subsubsection{功能号=2AH}
取日期,不要入口参数,返回的日期放在下述位置
\begin{enumerate}
    \item CX:年号(1980-2099),二进制
    \item DH:月号
    \item DL:日
    \item AL:星期
\end{enumerate}
\subsubsection{功能号=2BH}
设置日期。调用前CX,DX必须存有一个有效日期。调用后,若设置日期有效,则AL=00,若设置日期无效,则AL=0FFH.
\begin{lstlisting}
    MOV CX,1999;1999年(存为3CFH)
    MOV DH,04;4月
    MOV DL,15;15日
    MOV AH,2BH 
    INT 21H
\end{lstlisting}
\subsubsection{功能号=2CH}
取时间,不要入口参数,返回值在CX,DX中
\begin{enumerate}
    \item CH:小时
    \item CL:分钟
    \item DH:秒
    \item DL:1/100秒
\end{enumerate}
\subsubsection{功能号=2DH}
设置时间。调用前CX和DX中存放需设置的时间值,格式同2CH号调用。调用后AL的取值也同2CH号调用。
\begin{lstlisting}
    MOV CX,0824H;时间8:36
    MOV DX,1532H;21.5秒(15H=21,32H=50),50/100=0.5秒
    MOV AH,2DH 
    INT 21H
\end{lstlisting}
\subsection{返回操作系统}
功能号4CH,调用时可以强行返回DOS。一般用于程序末尾。