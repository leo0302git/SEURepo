\chapter{绪论}
\section{信号的概念}
广义地说,信号是随时间变化的某种物理量。信号可以表示为一个时间的函数。一般可以有几种分类
\subsection{确定信号和随机信号}
确定信号:信号是一个确定的时间函数,给定某一个时间值,就可以确定一个相应的函数值。

随机信号:当给定一个时间值时,其函数值不确定,只能知道此信号取某一数值的概率。工程中遇到的绝大部分信号都是随机信号
\subsection{连续信号与离散信号}
连续信号:如果在某一时间间隔内,对于一切时间值,除了若干不连续点以外,该函数都可以给出确定的函数值,则这信号就称为连续信号。
其中,在选取的时间零点以前值为0的函数,称为有始函数。

{\color{red}信号与函数近义但不等同。连续信号不一定是连续函数,连续信号允许不连续点(断点)的存在}

离散信号:离散信号的时间函数只在某些不连续的时间值上给出函数值。
\subsection{周期信号与非周期信号}
周期信号:对于任意时间点,都满足
\begin{equation}
    f(t)=f(t+T)
\end{equation}
其中T称为信号的周期。在实际生活中不存在严格的周期信号,但在本书中我们仍然讨论它。
\subsection{能量信号与功率信号}
假设信号是一个电路中的电流或电压,作用在$1\Omega$的电阻上,则在一定时间间隔中,负载电阻消耗的能量为
\begin{equation}
    W(t_1,t_2)=\int_{t_1}^{t_2}f^2(t)dt
\end{equation}
将时间间隔取无穷大,得到信号在整个时间区间上的能量
\begin{equation}
    W=\lim_{T\to \infty}\int_{-T}^{T}f^2(t)dt
\end{equation}
同理可以定义信号的平均功率
\begin{equation}
    P=\lim_{T\to \infty}\dfrac{1}{2T}\int_{-T}^{T}f^2(t)dt
\end{equation}
能量信号:能量为非零的有限值的信号.如脉冲信号

功率信号:平均功率为非零的有限值的信号。如周期信号

{\color{blue}{能量信号与功率信号是互斥的;存在于无限时间内的非周期信号可以是能量信号也可以是功率信号}}
\section{信号的简单处理}
相加与相乘、信号的延时、尺度变换和反褶。尺度变换的比例因子可以看成是磁带的播放速度,大于一时是加快,小于零时是倒放。
\section{系统的概念}
按照系统的特性,可以有如下的分类
\begin{enumerate}
    \item 线性系统和非线性系统:同时具有齐次性和叠加性的系统。定义可以扩充到“增量线性系统”
    \item 非时变系统和时变系统:系统的性质不随时间变化
    \item 连续时间系统和离散时间系统:根据处理的信号类型来分。在实际工作中往往是混合系统,两者皆有
    \item 因果系统和非因果系统:响应出现于激励之后的系统,这也是物理上能实现的系统
    \item 其他分类:集总参数/分布参数系统、有源/无源系统、即时/动态系统
\end{enumerate}